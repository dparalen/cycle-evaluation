\documentclass[12pt,oneside,draft]{fithesis}
\usepackage[plainpages=false, pdfpagelabels]{hyperref}
%\usepackage{amssym}
\usepackage{amsfonts}
\usepackage{amsmath}
\usepackage{amsthm}
\usepackage{graphics}
\usepackage{paralist}

%
\newcommand{\ltl}{\textsc{ltl}~}
\newcommand{\Next}{\emph{Next}~}
\newcommand{\mM}{\mathcal{M}}
\newcommand{\mD}{\mathcal{D}}
\newcommand{\mC}{\mathcal{C}}
\newcommand{\mI}{\mathcal{I}}
\newcommand{\mE}{\mathcal{E}}
\newcommand{\mS}{\mathcal{S}}
\newcommand{\mReal}{\mathbb{R}}
\newcommand{\mNatural}{\mathbb{N}}
\newcommand{\mTime}{\mathbb{T}}
\newcommand{\dStateVariables}{$Y'=\{y'_0,y'_1,\dotsc,y'_n\},n\in\mNatural$}
\newcommand{\dAtomicPropositionsI}{$AP_i=\{a_{i,0},a_{i,1},\dotsc,a_{i,m}\},m\in\mNatural$}
\newcommand{\bF}{\mathbf{F}}
\newcommand{\bG}{\mathbf{G}}
\newcommand{\bX}{\mathbf{X}}
\newcommand{\bU}{\mathbf{U}}
\newcommand{\bR}{\mathbf{R}}
\newcommand{\bA}{\mathbf{A}}
\newcommand{\bE}{\mathbf{E}}
%
\newtheorem{mydef}{Definition}
\newtheorem{mylemma}{Lemma}
%
\hyphenation{Kri-p-ke}
\hyphenation{boun-ded}
%
\thesistitle{Simulation Analysis of Large-scale Dynamic Systems}
\thesissubtitle{Bachelors Thesis}
\thesisstudent{Milan Kov\'{a}\v{c}ik}
\thesiswoman{false}
\thesisfaculty{fi}
\thesisyear{Fall 2011}
\thesislang{en}
\thesisadvisor{David \v{S}afr\'{a}nek}
%

\begin{document}

\FrontMatter
\ThesisTitlePage

\begin{ThesisDeclaration}
\DeclarationText
\AdvisorName
\end{ThesisDeclaration}

%\begin{ThesisThanks}
%\end{ThesisThanks}

%\begin{ThesisAbstract}
%\end{ThesisAbstract}

%\begin{ThesisKeyWords}
%\end{ThesisKeyWords}

\MainMatter
\tableofcontents
\chapter{Introduction}
%\section{Framework Description}
The framework main task is to select traces starting at the seeds of
initial conditions that fulfil given \ltl specification over the time
period provided. Therefore two main components are identified:
\em simulation and model checking\rm{}.
In the framework, the output of the simulation is processed by the
model checking component to verify the LTL formula validity of
a particular trace. Having verified all traces of the initial conditions
set, the framework marks the seeds fulfiling the formula to provide
the user with an output. User then could refine her initial conditions
specification for further system evaluation. In this fashion, the user
might iterate till a~subjective satisfaction with the system evaluation
was reached.

In such a~session, the framework provides simulation traces of the
system observed over the time specified. The simulation being rather
straightforward, a~common integrator -- such as provided in the
\textsc{lsoda} \cite{lsoda} package -- may be used.
An output in this case is an approximation of a~continuous trajectory
of the dynamic system.

\chapter{Background}
Considering the model checking and simulation topics, this section
introduces necessary concepts utilized troughout the text.

\section{Trace}
Here, a~trace $\pi$ represents result of either a~dynamic system
evolution over a~range of time\cite{sven}\cite{pospisil} approximated by
a~numeric method or a~result of any other kind of experiment providing
these conditions are met:
\begin{inparaenum}[\itshape a\upshape{})]
	\item trace points are vectors of state variable values from
		$\mReal$;
	\item in each point, a~temporal value from $\mReal^+$is present
		as the first field; and
	\item points can be ordered by their temporal value in a~strictly
		monotonic fashion.
\end{inparaenum}
\begin{mydef}[Trace]
Let $n\in\mNatural^+$ be the number of state variables of a~system.
A~sequence $\pi=p_0,p_1,\dotsc$ of tuples $p\in\mReal^+\times\mReal^n$
is caled a~trace. A~single element $p_i=\pi(i)$ of a~trace $\pi$ is
called a~point.
%The first point in a~trace $s_0=\pi(0)$ is called
%a~seed and a~trace starting with a~seed $s_0$ may be denoted as
%$\pi_{s_0}$.
A suffix of points $p_i,p_{i+1},\dotsc$ of a~trace is
denoted as $\pi^i$. The first field of the~point vector is called
a~time stamp.
A~fact that $p_i$ predeceases $p_{i+1}$ in a~trace is denoted as
$p_i\rightarrow p_{i+1}$. Trace points can be ordered in a~strictly
monotonic fashion based on their times tamps:
$p_i\rightarrow p_j\overset{def}{\iff} p_i(0)<p_j(0);i,j\in\mNatural_0$.
\end{mydef}

To be able to access the temporal field of a~trace point,
a~time projection function $\tau$ is defined.
\begin{mydef}[Time Projection]
Let $\pi$ be a~trace and $n\in\mNatural^+$ the number of state
variables trace $\pi$ involves. A~function
$\tau:\mReal^+\times\mReal^n\rightarrow\mReal^+$ is called a~time
projection and, given a~point $p\in\pi$, it provides the time stamp
$\tau(p)$ of that point.

Naturally, the definition can be extended to a trace as well;
$\hat{\tau}(\pi)=\tau(p_0),\tau(p_1),\dotsc;p_i\in\pi$
is then a~strictly monotonic sequence of relevant time stamps.
\end{mydef}

As far as dynamic systems are concerned, their evolution approximation
obtained as a~result of a~numerical method is always \emph{discrete and
finite}. The same holds for sampled real-life system experiments.

In case an infinitely long discrete trace comprises only a~finite number
of distinct points, it contains a~cycle. Such a~situation might be
denoted as follows\cite{biere}.
\begin{mydef}[K-Loop]
For $l\leq k;l,k\in\mNatural_0$ a~trace $\pi$ is called a~$(k,l)$-loop
if $\pi(k)\rightarrow\pi(l)$ and
$\pi=p_0,\dotsc,\overline{p_{l},\dotsc,p_{k},}$
A~$(k,l)$-loop may be called simply a~$k$-loop as well. 
\end{mydef}
A~loop is a~natural property of many real-life systems and represents
either an equlibrium $l=k$ or a~limit
cycle $l<k$\cite{sven}.

\section{Trace Properties}
Here, few trace qualities are enumerated including a~brief
description \cite{sven}\cite{rizk}:
\begin{inparaenum}[\itshape a\upshape)]
	\item\emph{reachability} given a~seed, one asks whether a~trace
		manages to reach \emph{a~specified region};
	\item\emph{attractor} given a~sample of seeds, the question:
		``Is there a~point, a~cycle or a~set in the traces phase space
		that some traces reach and remain in?" may be thought of as an
		informal description of an attractor \cite{wiki-atractor};
	\item\emph{repellor} in a~dynamic system of a~ball being dropped on
		a~steep hill from certain hight, the hill top acts as a~repellor
		as hitting it so that the ball eventually rests on it is
		unlikelly to happen \cite{wiki-repellor}.
	\item\emph{inevitability} the system necessarily fulfils given
		specification;
	\item\emph{invariance} a~property is always true
	\item\emph{response} fulfiling a~specification $\varphi$ triggers
		validity of a~specification $\psi$.
\end{inparaenum}

What all these trace properties have in common is they can be expressed
as an \ltl formula with real-based constraints as atomic
propositions\cite{sven}. Moreover, a~trace can be verified against
an \ltl formula. Therefore, the \ltl semantics is introduced.

\subsection{LTL Semantics}
\subsubsection*{Kripke Structure}
A~Kripke structure is a~common and satisfying way of modeling a~system,
especially for the purpose of the model validation\cite{clarke}.
\begin{mydef}[Kripke Structure]
Let ${AP}$ be a set of atomic propositions.
A~Kripke structure\cite{clarke} $M$ over $AP$ is a~tuple
$M=(S, S_0, R, L)$ where
\begin{inparaenum}[\upshape a\itshape)]
	\item{}$S$ is~a finite set of states;
	\item{}$S_0\subseteq{}S$ is the set of initial states;
	\item{}$R\subseteq{}S\times{}S$ is a~total transition relation; and
	\item{}$L:S\rightarrow{}2^{AP}$ is a~function that labels each state
		with a subset of atomic propositions valid in that state.
\end{inparaenum}
\end{mydef}
In the text, real-constraint atomic propositions are considered.
An example of such a~constraint could be
$a\equiv x_i \geq c;c\in\mReal$.

\begin{mydef}[Kripke Structure Path]
A~path\cite{biere} $\rho_M=s_0,s_1,\dotsc$ starting at a~state
$s$ in the structure $M$ is an infinite sequence of states
$s_i, s = s_0$ such that
$\forall i \in \mNatural_0: (s_i, s_{i+1}) \in R$ which may be denoted
as $s_i\rightarrow s_{i+1}$, too. 
Furthermore, $\rho_M(i)=s_i$ is the $i$-th state on the path and
$\rho_M^i=s_i,s_{i+1},\dotsc$ denotes an endless suffix of states
starting at the position $i$.
\end{mydef}

\begin{mydef}[LTL Semantics]
Let $M$ be a~Kripke structure, $\rho_M$ path in $M$ and $\varphi$ be
an \textsc{ltl} formula. Then a relation $\rho_M\models\varphi$
($\varphi$ is valid along $\rho_M$) is defined as follows\cite{clarke}.
%\begin{inparaenum}[\upshape a\itshape)]
\begin{align}
	\rho_M\models p&\iff p\in L(\rho_M(0))\\
	\rho_M\models\neg p&\iff p\notin L(\rho_M(0))\\
	\rho_M\models \varphi\wedge\psi&\iff\rho_M\models\varphi\wedge
		\rho_M\models\psi\\
	\rho_M\models \varphi\vee\psi&\iff\rho_M\models\varphi\vee
		\rho_M\models\psi\\
	\rho_M\models\bG\varphi&\iff\forall i:\rho_M^i\models\varphi,
		i\in\mNatural\\
	\rho_M\models\bF\varphi&\iff\exists i:\rho_M^i\models\varphi,
		i\in\mNatural\\
	\rho_M\models\bX\varphi&\iff\rho_M^1\models\varphi\\
	\rho_M\models\varphi\bU\psi&\iff\exists i,i\in\mNatural_0:
		\rho_M^i\models\psi\wedge(\forall j,0\leq j\leq i:
			\rho_M^j\models\varphi)\\
	\rho_M\models\varphi\bR\psi&\iff\forall j,j\in\mNatural_0,
		\forall i,i<j:\rho_M^i\not\models\varphi\implies
			\rho_M^j\models\psi
\end{align}
%\end{inparaenum}
\end{mydef}

\begin{mydef}[Validity]
An \ltl formula $\varphi$ is universally valid in a~Kripke structure $M$
if $\rho_M\models\bA\varphi\iff\rho_M\models\varphi$ holds for all
paths wiht $\rho_M(0)\in S_0$.

An \ltl formula $\varphi$ is existentially valid in a~Kripke structure
$M$ if $\rho_M\models\bE\varphi\iff\rho_M\models\varphi$ holds for all
paths with $\rho_M(0)\in S_0$\cite{biere}.
\end{mydef}


Determining whether an \ltl formula $\varphi$ is existentially (resp.
universally) valid in a~given Kripke structure is called an
\emph{existential} (resp. an \emph{universal}) \emph{model checking}
problem\cite{biere}.

\subsection{Bounded Model Checking}
In case of bounded model checking one consideres a~prefix of the path
$\rho_M$, $k\in\mNatural$ symbols long that is sufficient as a~witness
in case of existential problem\cite{biere}. Intuitively, if all
possible prefixes are checked, the method is equal to common \ltl model
checking approaches. Thus, if \ltl is considered, there is an upper
limit $k=|2^{AP}|$ of the counterexample length\cite{biere}. However, in
special case of a~\emph{lasso-shaped} path, a~smaller bound $k=|AP^2|$
exists\cite{biere}. Below, the semantics of bounded model checking is
introduced.

\subsubsection*{Bounded Semantics}
The idea of finding a~counterexample lies in unrolling temporal
operators in a~formula $\varphi$ to get a~formula $\phi$ comprising
of first order logic operators and original atomic propositions
``attached'' to a~particular position in $\rho_M$.

In bounded model checking, one uses only first $k+1$ elements of
a~path $\rho_M$ to determine the validity of the formula along the
path\cite{biere}.

If a~path is a~$k$-loop, the standard semantics of \ltl is maintained --
for the definition, see again\cite{biere}. The $\models$ operator gets
an index $\models_k$ to denote the bound.

In case of an acyclic path, the $\models$ operator is redefined to
$\models_k^i$ where the $i$ means a~\emph{position in the prefix} to
consider. Moreover, the \ltl operator $\bG$ is always false as nothing
repeats in $\rho_M$ making it impossible to be decided. For semantics
of other \ltl operators over a~finite path see once more\cite{biere}.

As far as the translation from $\varphi$ to $\phi$ is considered,
only the acyclic part is introduced here for the
sake of brevity. Both the acyclic and loop definitions are due to
Biere et al\cite{biere}.

\begin{mydef}[LTL Formula Translation without a Loop]
For an LTL formula $\varphi$ and $i,j,k\in\mNatural$, with $i\leq k$ and
$p\in {AP}_\varphi$, ${AP}_\varphi$ being the set of atomic propositions
of $\varphi$
\begin{align}
	[[p]]_k^i&\rightarrow p(s_i)\\
	[[\neg p]]_k^i&\rightarrow\neg(p(s_i))\\
	[[\varphi\wedge\psi]]_k^i&\rightarrow
		[[\varphi]]_k^i\wedge[[\psi]]_k^i\\
	[[\varphi\vee\psi]]_k^i&\rightarrow
		[[\varphi]]_k^i\vee[[\psi]]_k^i\\
	[[\bX\varphi]]_k^i&\rightarrow[[\varphi]]_k^{i+1}\\
	[[\bG\varphi]]_k^i&\rightarrow\bot\\
	[[\bF\varphi]]_k^i&\rightarrow\bigvee_{j=i}^k[[\varphi]]_k^j\\
	[[\varphi\bU\psi]]_k^i&\rightarrow\bigvee_{j=i}^k\left(
		[[\psi]]_k^j\wedge\bigwedge_{n=i}^{j-i}[[\varphi]]_k^n
	\right)\\
	[[\varphi\bR\psi]]_k^i&\rightarrow\bigvee_{j=i}^k\left(
		[[\varphi]]_k^j\wedge\bigwedge_{n=i}^j[[\psi]]_n^k
	\right)
\end{align} 
\end{mydef}
However, the definition being recursive, it is very unefficient to
apply the $[[\cdot]]_k^i$ operator every time as the procedure would
make exponentially many operations with regard to $k$ in that case.
Instead, one has to utilize sub-formula sharing to improve the
performance to a~polynomial time\cite{biere}.

\chapter{Metodology}
The key feature is the equivalence of trace points based on atomic
propositions validity. For that purpose, the relation $\sim$ is defined.
It legitimates the filtering of a~trace. Interestingly,
a~filtered trace resembles very closely what was previously defined as
a kripke structure path. Therefore $\sim$ allows \ltl model checking
to be applied to traces. Moreover, it even justifies the $\bX$ operator
semantics over a~continuous domain.
\begin{mydef}[Point Equivalence]
Points $p,q\in\pi$ are equivalent in $\sim$ if and only if following
condition holds.
\begin{align}
	p\sim q	&\overset{def}{\iff}\forall a\in AP:
		\bigwedge_{r\in\Omega(p,q)}a(q)\iff a(r)\\
\Omega(p,q)&=\{r\mid\tau(r)\in\langle\tau(p),\tau(q)\rangle\}\cup
	\{r\mid\tau(r)\in\langle\tau(q),\tau(p)\rangle\}
%\end{equation}
\end{align}
\end{mydef}
In words, two points $p,q\in\pi$ are in relation $p\sim q$ if they
have got the same evaluation in all the atomic propositions and all the
points laying ``between'' them -- from the temporal point of view --
have got the same evaluation, too.
\begin{mylemma}
Point equivalence $\sim$ is a~proper equivalence.
\begin{proof}Given a~trace $\pi$ and points $p,q,r\in\pi$, following
properties of $\sim$ are fulfiled rendering it a~proper equivalence.
	\begin{inparaenum}
		\item{Reflexivity.} As $\Omega(r,r)=\{r\}$, reflexivity holds.
		\item{Symmetry.} Bullox\dots
		\item{Transitivity.} Froobnicate the barochonder.
	\end{inparaenum}
	\qedhere
\end{proof}
\end{mylemma}
\end{document}
