\documentclass[12pt,oneside,draft]{fithesis}
\usepackage[plainpages=false, pdfpagelabels]{hyperref}
%\usepackage{amssym}
\usepackage{amsfonts}
\usepackage{amsmath}
\usepackage{amsthm}
\usepackage{graphics}
\usepackage{paralist}

%
\newcommand{\ltl}{\textsc{ltl}~}
\newcommand{\Next}{\emph{Next}~}
\newcommand{\mM}{\mathcal{M}}
\newcommand{\mD}{\mathcal{D}}
\newcommand{\mC}{\mathcal{C}}
\newcommand{\mI}{\mathcal{I}}
\newcommand{\mE}{\mathcal{E}}
\newcommand{\mS}{\mathcal{S}}
\newcommand{\mReal}{\mathbb{R}}
\newcommand{\mNatural}{\mathbb{N}}
\newcommand{\mTime}{\mathbb{T}}
\newcommand{\dStateVariables}{$Y'=\{y'_0,y'_1,\dotsc,y'_n\}$}
\newcommand{\bF}{\mathbf{F}}
\newcommand{\bG}{\mathbf{G}}
\newcommand{\bX}{\mathbf{X}}
\newcommand{\bU}{\mathbf{U}}
\newcommand{\bR}{\mathbf{R}}
%
\newtheorem{mydef}{Definition}
%
\thesistitle{Simulation Analysis of Large-scale Dynamic Systems}
\thesissubtitle{Bachelors Thesis}
\thesisstudent{Milan Kov\'{a}\v{c}ik}
\thesiswoman{false}
\thesisfaculty{fi}
\thesisyear{Fall 2011}
\thesislang{en}
\thesisadvisor{David \v{S}afr\'{a}nek}
%

\begin{document}

\FrontMatter
\ThesisTitlePage

\begin{ThesisDeclaration}
\DeclarationText
\AdvisorName
\end{ThesisDeclaration}

%\begin{ThesisThanks}
%\end{ThesisThanks}

%\begin{ThesisAbstract}
%\end{ThesisAbstract}

%\begin{ThesisKeyWords}
%\end{ThesisKeyWords}

\MainMatter
\tableofcontents
\chapter{Background}
In this chapter, various terms and constructions used troughtout the
thesis are being defined\dots
\section{Dynamic Systems}
\subsection{Dynamic System Definition}
A~dynamic system -- in this context -- abstracts from
a~set of chemical reactions happening inside an~organizm \cite{sven}.
In particular, it describes how the \emph{speed} of these chemical
reactions depends on factors such as \emph{stechiometry} of their
reactants and other influences taking part in the reactions.
The time and dynamic equation definitions are taken from
\cite{pospisil}.
\subsubsection{Time}
\begin{mydef}[Time]
Let $\mTime$ be a~set with a~total linear-order relation ``$<$",
with the operator $\sigma$; $\forall t \in \mTime{}.\:\exists
\sigma{}(t) \in \mTime$:
	\begin{equation}
		\sigma{}(t) = inf\{s : t < s\}
	\end{equation}
and the operator $\varrho$; $\forall t \in \mTime{}.\:\exists
\varrho{}(t) \in \mTime$:
	\begin{equation}
		\varrho{}(t) = sup\{r : r < t\}
	\end{equation}
and with a~continuous, order-preserving function
	$\nu: \mTime \times \mTime \rightarrow \mReal$
with the property:
	\begin{equation}
		\nu{}(t1, t3) = \nu{}(t_1, t_2) + \nu{}(t_2, t_3)
	\end{equation}
$\mTime$ together with these shall be refered to as the \emph{Time}.
\end{mydef}
The function $\nu$ may be thought of as a~distance measurement.
The operators $\sigma$ and $\varrho$ give the \emph{next} and the
\emph{previous}
Time values respectivelly. The definition allows the Time
to have an ``end" or to comprise of discrete or even
a~finite number of points.

\begin{mydef}[Time Granularity]
Let $\mTime$ be the Time and $t \in \mTime$. Then a~function
$\mu: \mTime \rightarrow \mReal$
	\begin{equation}
		\mu{}(t) = \nu{}(\sigma{}(t), t)
	\end{equation} is called time granularity.
\end{mydef}
If a~continuous Time is considered,
$\forall{}t\in\mTime.\:\mu{}(t) = 0$.
A~discrete Time on the other hand may have many values of $\mu$
depending on how $\sigma$ and $\varrho$ are defined. In some situations,
even $\forall{}t\in\mTime.\:\mu{}(t) = c, c \in \mReal - \{0\}$ is
possible, meaning the Time is granular with a~constant step.

\subsubsection{Dynamic Equation}
\begin{mydef}[Dynamic Equation]
Let $y$ be a~state variable and $y'$ be its derivation with regard to
Time. Furthermore, let $t \in \mTime$ be a~point in the Time.
Then the function
	\begin{equation}
		y'(t)=f(t, influences)
	\end{equation}
denotes a~dynamic equation of the state variable $y$.
\end{mydef}
The influences within $f$ may represent mutual interactions of another
state variables as well as the variable $y$ -- in which case the
$y'$ contains a feedback loop -- with the state variable $y$.
Furthermore, the influences may express the effect of constants to
state variables and internal variables of the dynamic equation. In case
the influences do not depend on Time, the function is said to be
\emph{autonomous}.
For particular examples of the influences with regards to
chemical reactions see \cite{sven}.
As far as the codomain values type of the function is considered, a~real
is usual. Although, if a system of dynamic equations is considered, the
type is rather a tuple of real values.

\begin{mydef}[Seed]
Let $y$ be a~state variable and $t_0 \in \mTime$ a point in Time.
A~seed $s$ is the initial value of the variable $y$:
	\begin{equation}
		s=y(t_0)
	\end{equation}
\end{mydef}
In most of the cases, $t_0 = 0$ as relative time measurement usually
starts at that point. In the text, a~subscript at a~dynamic equation
reflects the seed considered: $y'_s$.

\begin{mydef}[Trace]
Let $y'$ be a~dynamic equation of a~state variable $y$. Furthermore,
let $s$ be a seed of $y$.
A~trace $\pi_s$ is a~sequence of values of $y_s'(t)$ in all poitns
of the Time considered and y initialized with the value of the seed $s$:
	\begin{equation}
		\pi_s = \{y'_s(t) : t \in \mTime\}
	\end{equation}
\end{mydef}
A~term \emph{trajectory} may be used in the text refering to the same.
Still, some might find trajectory providing a~continuous trace a~better
name.

\subsubsection{Dynamic Equations System}
\begin{mydef}[Dynamic Equations System]
Let $Y'=\{y'_i:1\leq{}i\leq{}n,\:n\in\mNatural\}$ be a~set of dynamic
equations of state variables $y_i$.
Furthermore, let $t\in\mTime$ be a~point in Time.
A~dynamic system is a~function
$\mD:E\times\mTime\rightarrow\mReal^n$ mapping
time values to tuples of particular state variable values:
	\begin{equation}
		\mD(Y',t)=(y'_{1}(t),y'_{2}(t),\dotsc,y'_{n}(t)).
	\end{equation}
\end{mydef}
In the text, a~short form \emph{dynamic system} may be used to denote
the same. Moreover, the first component $Y'$ is ommited in
the rest of the text for brevity. Thus if reffered to, a~system $\mD$
represents a~particular set of dynamic equations as well.

As far as the context is considered, a~dynamic system refers
to the change of concentrations of species $y_i$ in a~set of chemical
reactions with regards to time. These are expressed as the influences
of particular transition functions.

The dynamic system comprising of dynamic equations, it is natural to
extend the definition of the trace to fit the system as well.

\begin{mydef}[Dynamic System Seed]
Let $\mD$ be a~dynamic system comprising of a~set $Y'$ of dynamic
equations $Y'=\{y'_i:1\leq{}i\leq{}n,n\in\mNatural\}$.
Furthermore, let $t\in\mTime$ be a~point in Time.
A~tuple $s=(s_1,s_2,\dotsc,s_n)$ of seeds $s_i$ of particular
function $y'_i$ is a~dynamic system seed. It is applied to the functions
as follows:
	\begin{equation}
		\mD_s{}(t)=(y'_{1,s_1}(t),y'_{2,s_2}(t),\dotsc{},y'_{n,s_n}(t))
	\end{equation}
\end{mydef}
As well as for its equation counterpart, the seed of a dynamic system
$\mD$ is denoted as its subscritpt in the text: $\mD_s$.

\begin{mydef}[Dynamic System Trace]
Let $\mD$ be a~dynamic equation system comprising of dynamic equations
$Y'=\{y'_i:1\leq{}i\leq{}n,\:n\in\mNatural\}$. Furthermore, let
$t\in\mTime$ be a~point in Time and $s=(s_1,s_2,\dotsc,s_n)$ be
its~seed. A~sequence $\pi_s$ of tuples of $y'_{i,s_i}(t)$ values in all
points of the Time considered is a~system trace:
	\begin{equation}
		\pi_s=\{\mD_s{}(t): t\in\mTime\}
	\end{equation}
\end{mydef}

Further in the text, dynamic systems, dynamic system seeds as well as
dynamic system traces may be indexed to refer to a~particular component:
\begin{inparaenum}[\itshape a\upshape{})]
	\item $\mD_i$ refers to the function $y'_i$ of the system;
	\item $s_i$ refers to the i-th seed value within the seed tupple
		$s$;
	\item $\mD_{s,i}$ refers to the ``seeded" function $y'_{i,s_i}$; and
	\item $\pi_{s,i}$ refers to a trace of a~seeded function
		$y'_{i,s_i}$.
\end{inparaenum}

\subsection{Properties of Dynamic Systems}
One of the interesting properties of a~dynamic system trace
surely is its shape; as far as the dynamic functions remain continuous,
the trace represents \emph{a~contimuous trajectory}, too \cite{sven}.
This property further determines other trace qualities enumerated here
with a~brief description:
\begin{inparaenum}[\itshape a\upshape)]
	\item\emph{reachability} given a~seed, one asks whether a~trace
		manages to reach \emph{a~specified region} on time;
	\item\emph{attractor} given a~sample of seeds, the question:
		``Is there a~point, a~cycle or a~set in the traces phase space
		that some traces reach and remain in?" may be thought of as an
		informal description of an attractor \cite{wiki-atractor}; and
	\item\emph{repellor} in a~dynamic system of a~ball being dropped on
		a~steep hill from certain hight, the hill top acts as a~repellor
		as hitting it so that the ball eventually rests on it is
		unlikelly to happen \cite{wiki-repellor}.
\end{inparaenum}

A~dynamic system might actually contain multiple instances of both the
attractors and repellors -- a~multiple hill-tops and valleys landscape
for the ball to drop in. In case of points, they may be refered to as
\emph{equilibria}. However, limit-cycles are as well examples of an
attractor.
\subsubsection{Initial Conditions Problem}
A~problem of deciding whether, given a~seed and Time, a~trace has some
of these properties is called the \emph{initial conditions problem}
\cite{sven}.

\section{Linear Temporal Logic}
The Linear Temporal Logic -- \textsc{ltl} -- allows one to specify
properties of a~dynamic system trace \cite{sven} hence the prototype
utilizes it for this purpose. Following is a~brief
\textsc{ltl} specification.
\subsection{Bounded Semantics}
\subsubsection{Kripke Structure}
A~Kripke structure is a~common and satisfying way of modeling a~system,
especially for the purpose of the model validation\cite{clarke}.
\begin{mydef}[Kripke Structure]
Let ${AP}$ be a set of atomic propositions.
A~Kripke structure\cite{clarke} $M$ over $AP$ is a~tuple
$M=(S, S_0, R, L)$ where
\begin{inparaenum}[\upshape a\itshape)]
	\item{}$S$ is~a finite set of states;
	\item{}$S_0\subseteq{}S$ is the set of initial states;
	\item{}$R\subseteq{}S\times{}S$ is a~total transition relation; and
	\item{}$L:\rightarrow{}2^{AP}$ is a~function that labels each state
		with a subset of atomic propositions valid in that state.
\end{inparaenum}
\end{mydef}

\begin{mydef}[Kripke Structure Path]
A~path\cite{biere} $\rho_M=s_0,s_1,\dotsc$ starting at a~state
$s$ in the structure $M$ is an infinite sequence of states
$s_i, s = s_0$ such that
$\forall i \in \mNatural_0: (s_i, s_{i+1}) \in R$.
Furthermore, $\rho_M(i)=s_i$ is the $i$-th state on the path and
$\rho_M^i=s_i,s_{i+1},\dotsc$ denotes an endless suffix of states
starting at the position $i$.
\end{mydef}

\begin{mydef}[LTL Semantics]
Let $M$ be a~Kripke structure, $\rho_M$ path in $M$ and $\varphi$ be
an \textsc{ltl} formula. Then a relation $\rho_M\models\varphi$
($\varphi$ is valid along $\rho_M$) is defined as follows\cite{biere}.
%\begin{inparaenum}[\upshape a\itshape)]
\begin{enumerate}
	\item{$\rho_M\models p$} $\iff p\in L(\rho_M(0))$
	\item{$\rho_M\models\neg p$} $\iff p\notin L(\rho_M(0))$
	\item{$\rho_M\models \varphi\wedge\psi$}
		$\iff\rho_M\models\varphi$ and
		$\rho_M\models\psi$
	\item{$\rho_M\models \varphi\vee\psi$}
		$\iff\rho_M\models\varphi$ or
		$\rho_M\models\psi$
	\item{$\rho_M\models\bG\varphi$}
		$\iff\forall i: \rho_M^i\models\varphi$
	\item{$\rho_M\models\bF\varphi$}
		$\iff\exists i: \rho_M^i\models\varphi$
	\item{$\rho_M\models\bX\varphi$}
		$\iff\rho_M^1\models\varphi$
	\item{$\rho_M\models\varphi\bU\psi$} $\iff$
		$\exists i: \rho_M^i\models\psi \wedge\forall j, j<i: \rho_M^j$
	\item{$\rho_M\models\varphi\bR\psi$} $\iff$
$\forall i,\forall j,j<i:\rho_M^j\not\models\varphi\implies\rho_M^i\models\psi$
\end{enumerate}
$i,j\in\mNatural_0$.
%\end{inparaenum}
\end{mydef}

Given a~limit of a~prefix lenght $\rho$ can be decided to model
a true or false\dots
%Boundedness of the semantics means that -- instead of a~common
%graph-theoretical approach such as automata based verification
%\cite{clarke} -- 

%Even though the path closely resembles the dynamic system trace both in
%its definition and symbol, it should be clear from the context to what
%definition the author refers. Still, the goal is to be able to use
%a~system trace in places where one would rather expect a~kripke
%structure path. Moreover, when a dynamic system is being refered, it is
%useful to define a~way atomic propositions describe system state
%variables.

\subsection{LTL Modeling}
To be able to connect atomic propositions with the kripke structure
definition, following constructions are used. The goal is to be able
to check whether a~dynamic system trace $\pi$ satisfies a~condition
expressed as an \textsc{ltl} property.

\subsubsection{Atomic Propositions}
\begin{mydef}[Atomic Proposition]
Given a~set of state variables  $Y'=\{y'_0,y'_1,\dotsc,y'_n\}$ of
a~dynamic system $\mD$, a~domain of the variables $\mReal$ and
a~set of operators $\chi=\{>,\geq,<,\leq,=\}$, a~tuple
$a=(y'_i\odot r)\in Y'\times\chi\times\mReal$ is called an atomic
proposition.

$AP$ denotes a~set of all atomic propositions.
If indexed, $a_{i,j}\in AP_i$ refers to an atomic proposition related to
particular state variable.
\end{mydef}


\begin{mydef}[AP Encoding]
Given a~set $Y'=\{y'_0,y'_1,\dotsc,y'_n\}$ of state variables of
a~dynamic system $\mD$ and an atomic proposition $a_i\in AP_i$ of
a~particular state variable $y'_i$, a~function
$l:AP_i\rightarrow S_i$ is an encoding assigning particular state
variable atomic propositions $a_{i,j}$ their
kripke structure state representation $s_{i,j}$ as follows.
Let $a_{i,j}=(y'_i\odot r), a_{i,j}\in AP_i$. Depending on the
operator $\odot$ type, appropriate (interval) values are assigned
to $l(a_{i,j})=s_{i,j}$
	\begin{enumerate}
		\item{`$=$'} $\langle r,r\rangle$
		\item{`$<$'}
			\begin{enumerate}
				\item{$(-\infty,r)$} $r$ is the lowest constant
					amongst $AP_i$
				\item{$(r',r)$} in case there is $r'<r$ of the same
					state variable
				\item{$\langle r,\infty)$} $r$ the biggest constant
			\end{enumerate}
		\item{`$>$'}
			\begin{enumerate}
				\item{$(-\infty,r\rangle$} r is the lowest constant
				\item{$(r,r')$} in case there is $r'>r$
				\item{$(r,\infty)$} in case $r$ is the biggest constant
			\end{enumerate}
		\item{`$\leq$'}
			\begin{enumerate}
				\item{$(-\infty,r\rangle$} $r$ the lowest constant
				\item{$(r', r\rangle$} in case there is $r'<r$
				\item{$(r,\infty)$} in case $r$ is the biggest constant
			\end{enumerate}
		\item{`$\geq$'}
			\begin{enumerate}
				\item{$(-\infty,r)$} $r$ the lowest constant
				\item{$\langle r, r')$} if there is $r'>r$
				\item{$\langle r, \infty)$} if $r$ is the biggest
						constant
			\end{enumerate}
	\end{enumerate}
\end{mydef}
Thus a~set of kripke states $S_i$ of a~state variable $y'_i$ might
contain these items $S_i=\{(-\infty,7\rangle,(7,\infty)\}$ if
its set of atomic propositions was $AP_i=\{(y'_i \leq 7)\}$.

\begin{mydef}[AP Evaluation]
Given an atomic proposition $a=(y'_i\odot r)$, a~dynamic system $\mD$
consisting of \dStateVariables state variables, a~point in Time
$t\in\mTime$ the system operates in and a~seed, a~function
$\epsilon_{\mD_s}:\mTime\times AP\rightarrow\{\bot,\top\}$
asinging either true or false value to $a$ based on common semantics of
the operator $\odot$ evaluation over the real numbers domain
is called atomic proposition evaluation function.
\end{mydef}


\chapter{Methodology}
\section{Framework Description}
The framework main task is to select traces starting at the seeds of initial conditions
that fulfil given \ltl specification over the time period provided.
Therefore two main components are identified: \em simulation and model checking\rm{}.
In the framework, the output of the simulation is processed by the
model checking component to verify the LTL formula validity of a particular
trace. Having verified all traces of the initial conditions set, the framework
marks the seeds fulfiling the formula to provide the user with an output. 
User then could refine her initial conditions specification for further
system evaluation. In this fashion, the user might iterate till a~subjective satisfaction
with the system evaluation was reached.

In such a~session, the framework provides simulation traces of the system observed over
the time specified. The simulation being rather straightforward, a~common integrator -- such as provided in the \textsc{lsoda}
\cite{lsoda} package -- may be used.
An output in this case is an approximation of a~continuous solution of the dynamic system.

Even though the approximation is discrete at the finest grain, such a~detail isn't reasonable
for the model checking part of the framework. Especially, the \Next operator semantics doesn't make sense over the
(aproximated) continuous domain. If one selects $x_1 = Next(x_0)$, many system points can stil be inserted
inbetween $x_0$ and $x_1$ as other options for $x_1$ making the original $x_1$ point selection invalid.
This happens e.g.~when the precision of the integration changes \cite{integration}.
As a~consequence, even on a~single trace, an~\ltl formula evaluation might gave both the positive and the negative result
based just on the precision the sysetem was integrated with. 
%(Thus a discrete and deterministic trace points assignment mechanism is required.)

However, considering the \em constants \rm present within the formula as parts of the real \em constrains \rm put on the 
trajectory to be fulfiled, a~partitioning of the trace naturarlly emerges. Adding time to the consideration, one is able to
observe particular constants--boundaries being crossed by the system as it evolves. \em Focusing just on the moment of
the boundary-crossing, one gains a~series of discrete events the system exhibits\rm{}.

Such an~observation resembles what one might call \em a~filter \rm and indeed a~filtering sub-component of the framework
can be identified this way. Moreover, as the same constants would be used for an~\ltl b\"{u}chi automaton representation of the formula, 
crossing any of them would make the representation change its internal state as well.
Thus, considering the state change -- \Next semantics, the construction seems valid from the model checking point of view.

The main focus of the tool being seeds marking with the \ltl evaluation result, one could discard the original simulation
data in favor of the discrete points.
Doing so would have the benefit of memory usage relaxation which, as the framework evaluation suggests, would be quite substantial.

\section{Filter}
% TODO: redefine with both the operators <, > evaluation pair
% else the construction doesn't work
As already mentioned, if the focus is put on the sole moment of boundary
crossing, discrete states are obtained from a~continuous system.
Following are definitions that describe the filtering.
\begin{mydef}[State Change]
Given a~dynamic system $\mD$, its seed $s$ and an \ltl formula $\varphi$
let $\varphi$ contain a set of constants $C$.
Furthermore, let $t_0,t\in\mTime;t_0<t$ be two points in time.
A~state change relation
$\delta\subseteq\mTime\times\mTime$ with regards to $\mD_s$
and $\varphi$ is a set:
	\begin{equation}
\delta_{\varphi_s} = \{(t_0,t):\exists{}c_i\in{}C,\mD_{s,i}(t_0)>c_i\neq
\mD_{s,i}(t)>c_i\}
	\end{equation}
\end{mydef}
To denote a~system changed its state with regards to $\varphi$, the
notation $(t_0,t)\in\delta_{\varphi_s}$ can be used.
In fact, the $<$ relation may be used in $\delta$ definition as well --
the purpose is only to obtain a~true--false value.

\begin{mydef}[Filtered Time Operators]
Let $\mTime$ be the Time and $t\in\mTime$ a~point in Time.
Let $\mD$ be a~dynamic system with its seed $s$.
Furthermore, let $\varphi$ be an \textsc{ltl} formula and
$\delta_{\varphi_s}$ a~state change relation with regards to $\mD_s$.
A~filtered operator $\sigma_{\varphi_s}:\mTime\rightarrow\mTime$ is defined
as follows:
	\begin{equation}
		\sigma_{\varphi_s}(t) = inf\{u:(t,u)\in\delta_{\varphi_s}\}
	\end{equation}
A~filtered operator $\varrho_{\varphi_s}:\mTime\rightarrow\mTime$ is defined
as follows:
	\begin{equation}
		\varrho_{\varphi_s}(t) = sup\{u:(u,t)\in\delta_{\varphi_s}\}
	\end{equation}
\end{mydef}

\begin{mydef}[Filtered Time]
Let $\mD$ be a~dynamic system with a~seed $s$. Let $t_0\in\mTime$
be a~point in Time. Furthermore, let $\varphi$ be an \textsc{ltl}
formula. A~set $\mTime_{\varphi_s}$ of all time points from $\mTime$ such
that they are in the $\delta_{\varphi_s}$ state change relation is
a~$\varphi$-filtered Time of a~seeded system $\mD_s$:
$$t_0\in\mTime_{\varphi_s}$$
$$t\in\mTime_{\varphi_s}\Rightarrow\sigma_{\varphi_s}(t)\in\mTime_{\varphi_s}$$
\end{mydef}

For the sake of completness, a~filtered trace notion is introduced as
well.
\begin{mydef}[Filtered Trace]
Let $\mD$ be a~dynamic system wiht a~seed $s$. Let $\varphi$ be an
\textsc{ltl} formula and $\mTime_{\varphi_s}$ be a~$\mD_s$ Time filtered
with regards to $\varphi$. A~filtered trace $\pi_{\varphi_s}$ is a~set of
values of the system in all points of the filtered time:
	\begin{equation}
		\pi_{\varphi_s}=\{\mD_s(t):t\in\mTime_{\varphi_s}\}
	\end{equation}
\end{mydef}
%\begin{mydef}[Filter]
%
%\end{mydef}
% list of figures
% list of tables
\bibliographystyle{plain}
%\bibliography{thesis.bib}
%

\end{document}  
