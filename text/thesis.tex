\documentclass[11pt]{fithesis}
\usepackage[plainpages=false, pdfpagelabels]{hyperref}
%\usepackage{amssym}
\usepackage{amsfonts}
\usepackage{amsmath}
\usepackage{amsthm}
\usepackage{graphics}

%
\newcommand{\ltl}{\textsc{ltl}~}
\newcommand{\Next}{\emph{Next}~}
\newcommand{\mM}{\mathcal{M}}
\newcommand{\mD}{\mathcal{D}}
\newcommand{\mC}{\mathcal{C}}
\newcommand{\mI}{\mathcal{I}}
\newcommand{\mReal}{\mathbb{R}}
\newcommand{\mNatural}{\mathbb{N}}
%
\newtheorem{mydef}{Definition}
%
\thesistitle{Simulation Analysis of Large-scale Dynamic Systems}
\thesissubtitle{Bachelors Thesis}
\thesisstudent{Milan Kov\'{a}\v{c}ik}
\thesiswoman{false}
\thesisfaculty{fi}
\thesisyear{Fall 2011}
\thesislang{en}
\thesisadvisor{David \v{S}afr\'{a}nek}
%

\begin{document}

\FrontMatter
\ThesisTitlePage

\begin{ThesisDeclaration}
\DeclarationText
\AdvisorName
\end{ThesisDeclaration}

%\begin{ThesisThanks}
%\end{ThesisThanks}

%\begin{ThesisAbstract}
%\end{ThesisAbstract}

%\begin{ThesisKeyWords}
%\end{ThesisKeyWords}

\MainMatter
\tableofcontents
\chapter{Methodology}
\section{Framework Description}
The framework main task is to select traces starting at the seeds of initial conditions
that fulfil given \ltl specification over the time period provided.
Therefore two main components are identified: \em simulation and model checking\rm{}.
In the framework, the output of the simulation is processed by the
model checking component to verify the LTL formula validity of a particular
trace. Having verified all traces of the initial conditions set, the framework
marks the seeds fulfiling the formula to provide the user with an output. 
User then could refine her initial conditions specification for further
system evaluation. In this fashion, the user might iterate till a~subjective satisfaction
with the system evaluation was reached.

In such a~session, the framework provides simulation traces of the system observed over
the time specified. The simulation being rather straightforward, a~common integrator -- such as provided in the \textsc{lsoda}
\cite{lsoda} package -- may be used.
An output in this case is an approximation of a~continuous solution of the dynamic system.

Even though the approximation is discrete at the finest grain, such a~detail isn't reasonable
for the model checking part of the framework. Especially, the \Next operator semantics doesn't make sense over the
(aproximated) continuous domain. If one selects $x_1 = Next(x_0)$, many system points can stil be inserted
inbetween $x_0$ and $x_1$ as other options for $x_1$ making the original $x_1$ point selection invalid.
This happens e.g.~when the precision of the integration changes \cite{integration}.
As a~consequence, even on a~single trace, an~\ltl formula evaluation might gave both the positive and the negative result
based just on the precision the sysetem was integrated with. 
%(Thus a discrete and deterministic trace points assignment mechanism is required.)

However, considering the \em constants \rm present within the formula as parts of the real \em constrains \rm put on the 
trajectory to be fulfiled, a~partitioning of the trace naturarlly emerges. Adding time to the consideration, one is able to
observe particular constants--boundaries being crossed by the system as it evolves. \em Focusing just on the moment of
the boundary-crossing, one gains a~series of discrete events the system exhibits\rm{}.

Such an~observation resembles what one might call \em a~filter \rm and indeed a~filtering sub-component of the framework
can be identified this way. Moreover, as the same constants would be used for an~\ltl b\"{u}chi automaton representation of the formula, 
crossing any of them would make the representation change its internal state as well.
Thus, considering the state change -- \Next semantics, the construction seems valid from the model checking point of view.

The main focus of the tool being seeds marking with the \ltl evaluation result, one could discard the original simulation
data in favor of the discrete points.
Doing so would have the benefit of memory usage relaxation which, as the framework evaluation suggests, would be quite substantial.

\section{Filter}
As already mentioned, if the focus is put on the sole moment of boundary
crossing, discrete states are obtained from a~continuous system.
Following are definitions that describe the filtering.
\begin{mydef}[State Change]
Given a~dynamic system $\mD$, a~set of dimension indexes $Dim$, an~\ltl formula $\varphi$ and an~initial
condition $i \in \mI$, let $\varphi$ contain
$\mC\,\subset\,\mReal, |\mC|\,\in\,\mNatural$, a~set of constants related to
particular dimension describing the constrains. Furthermore, let
$t_0, t \in \mReal^{+}, t_0~<~t$, two points in time.
$\mD(i, t)$ exhibits a~State~Change~$\sigma$ with regards to $\mD(i,~t_0)$ iff
$$\exists~d~\in~Dim, c_d~\in~\mC.\quad \mD(i,~t_0)[d]~>~c_d~\neq~\mD(i,~t)[d]~>~c_d.$$
Here $\mD(i,~t_0)[d]$ denotes value of the dynamic system expressed in time
$t_0$ in dimension $d$ staring at initial condition $i$; $c_d$
is a~constant related to dimension $d$.
\end{mydef}
To denote a~system changed its state, the notation
$\mD(i,~t_0)~\sigma~\mD(i,~t)$ can be used.
In fact, the $<$ relation may be used in $\sigma$ definition as well --
the purpose is only to obtain a~true--false value.
 
%\begin{mydef}[Filter]
%
%\end{mydef}
% list of figures
% list of tables
\bibliographystyle{plain}
%\bibliography{thesis.bib}
%

\end{document}  
